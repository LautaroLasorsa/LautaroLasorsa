\documentclass[10pt]{article}
\usepackage{graphicx} % Required for inserting images
\setlength{\parindent}{0pt}
\usepackage{hyperref}
\usepackage{enumitem}
\usepackage[utf8]{inputenc} 
\usepackage[T1]{fontenc}
\usepackage[brazil]{babel}
\usepackage{lipsum}
\usepackage[left=1.06cm,top=1.7cm,right=1.06cm,bottom=0.49cm]{geometry}




%by: Aline R. Antunes

\begin{document}
\begin{center}
    \textbf{Lautaro Lasorsa}\\ 
    \hrulefill
\end{center}

\begin{center}
    lautarolasorsa@gmail.com \textbullet \href{https://www.linkedin.com/in/lautaro-lasorsa/}{LinkedIn: lautaro-lasorsa} \textbullet \href{https://github.com/LautaroLasorsa}{GitHub: LautaroLasorsa}
\end{center}

\vspace{0.5pt}

\begin{center}
    \textbf{Education}
\end{center}
\textbf{Universidad de Buenos Aires (UBA)} 

Data Science Licenciature (Bachelor + Master): 9,09 \hfill March 2019 - December 24 

Relevant Knowledge: Algebra, statistics, interpretable models, cloud providers, operative research, econometrics, Python.



\textbf{Universidad Nacional de La Matanza (UNLaM)} 

Engineering in Informatics (Bachelor + Master) : 8,46 (partial) \hfill March 2019 - December 2027 (expected)

Relevant Knowledge: Bash, good software engineering practices, cryptography, software requirements, C++, Bash.



\begin{center}
    \textbf{Most relevant experience}
\end{center}

\textbf{Autoscheduler.AI} \hfill (Remote)\\
\textbf{Machine Learning Engineer} \hfill July 2025 - Present
\begin{itemize}[noitemsep, topsep=0pt, partopsep=0pt, parsep=0pt]
    \item Used langchain and langgraph to develop an AI-powered assistant.
    \item Developed the tools to integrate the assistant with the database and make detailed reports and plots.
\end{itemize}

\vspace{8pt}

\textbf{Mutt Data} \hfill (Remote)\\
\textbf{Ssr Data Developer} \hfill February 2025 - June 2025
\begin{itemize}[noitemsep, topsep=0pt, partopsep=0pt, parsep=0pt]
    \item Used Typesense to develope a real time search engine.
    \item Define metrics and objectives with the client.
\end{itemize}

\vspace{8pt}

\textbf{Composable Finance} \hfill (Remote)\\
\textbf{Cryptography Consultant} \hfill April 2023 - July 2024
\begin{itemize}[noitemsep, topsep=0pt, partopsep=0pt, parsep=0pt]
    \item Developed solutions using Zero Knowledge (ZK) over Gnark.
    \item Implemented the verification of digital signatures over the curve ED-25519 emulating the arithmetic.
    \item Learned the technology during the job.
    \item Also developed a solution to optimize transactions using graph algorithms and Python.
\end{itemize}

\vspace{8pt}

\textbf{Universidad de Buenos Aires (UBA)}\\
\textbf{Head Class Assistant (JTP)} \hfill March 2025 - Present
\begin{itemize}[noitemsep, topsep=0pt, partopsep=0pt, parsep=0pt]
    \item Directed practical classes about Algorithms and Data Structures.
    \item Designed the midterm exams.
\end{itemize}

\vspace{8pt}

\textbf{Universidad Nacional de La Matanza (UNLaM)}\\
\textbf{Class Assistant} \hfill March 2020 - Present
\begin{itemize}[noitemsep, topsep=0pt, partopsep=0pt, parsep=0pt]
    \item Taught classes to students from high schools (2020 and 2021) and from the university (2022 in ahead) about competitive programming
    \item Developed and used class material in C++, Python and Kotlin.
    \item 11 teams that participated in the course presented at the Argentina's Programming Tournament (TAP) between 2022 and 2024.
\end{itemize}

\vspace{8pt}

\textbf{Competitive Programming} 
\begin{itemize}[noitemsep, topsep=0pt, partopsep=0pt, parsep=0pt]
   \item \href{https://icpc.global/ICPCID/DV1XEVUDPG8J}{3 Participations at ICPC} (International Collegiate Programming Contest) World Final: 2020 (position 22), 2023 (13 - Bronze Medal), 2024 (17). Latin America Regional Champion all the times.
   \item 3 Participations at IOI (International Olympics in Informatic): 2017, 2018 (Bronce Medal, top 50\%) y 2019
    \item Used several Computing Science topics (Algorithms and Data Structures, Number Theory, Computational Geometry, etc.)
    \item Presented in 4 editions of the Training Camp Argentina on the topics \textit{Segment Tree}, \textit{SQRT Decomposition}, \textit{Centroid Decomposition and Heavy Light Decomposition} and \textit{Flow Algorithms}.
    \item Developed a \href{https://github.com/LautaroLasorsa/competitive-programming-suite}{repository} with useful commands for competitive programming.
   
\end{itemize}

\begin{center}
    \textbf{Other interests and skills}
\end{center}

\textbf{Languages:} English (\href{https://cert.efset.org/Pymcaz}{EF SET 76/100 C2}).

\textbf{Scientific publications:} Since 2021 I am a member in a research group at UNLaM. We had published a 3 papers (
\href{https://github.com/LautaroLasorsa/CONAIISI-2023}{1}, \href{https://github.com/LautaroLasorsa/CONAIISI-2024}{2},
\href{https://github.com/carlucho1/CONAIISI-2024-2}{3}
) about AI applied to IoT. 
 

% \textbf{Laboratory:} List scientific / research lab techniques or tools [If Applicable]

% \textbf{Interests:} List activities you enjoy that may spark interview conversation




\end{document}
